\chapter{Практическая часть}

\section{Задание}

Создать базу знаний <<Собственники>>, дополнив (и минимально изменив) базу знаний, хранящую знания (лаб. 7), знаниями о дополнительной собственности владельца:
\begin{itemize}
	\item <<Телефонный справочник>>: Фамилия, №тел, Адрес~--- структура (Город, Улица, №дома, №кв),
	\item <<Автомобили>>: Фамилия\_владельца, Марка, Цвет, Стоимость и др.,
	\item <<Вкладчики банков>>: Фамилия, Банк, Счет, Сумма и др.
\end{itemize}

Преобразовать знания об автомобиле к форме знаний о собственности.
Вид собственности (кроме автомобиля):
\begin{itemize}
	\item Строение: стоимость и другие его характеристики;
	\item Участок: стоимость и другие его характеристики;
	\item Водный\_транспорт: стоимость и другие его характеристики.
\end{itemize}

Описать и использовать вариантный домен \textbf{Собственность}.
Владелец может иметь, но только один объект каждого вида собственности (это касается и автомобиля), или не иметь некоторых видов собственности.
Используя конъюнктивное правило и разные формы задания одного вопроса (пояснять, для какого задания~--- какой вопрос), обеспечить возможность поиска:
\begin{enumerate}
	\item названий всех объектов собственности заданного субъекта,
	\item названий и стоимости всех объектов собственности заданного субъекта,
	\item разработать правило, позволяющее найти суммарную стоимость всех объектов собственности заданного субъекта.
\end{enumerate}

Для 2-го пункта и одной фамилии составить таблицу, отражающую конкретный порядок работы системы, с объяснениями порядка работы и особенностей использования доменов (указать конкретные T1 и T2 и полную подстановку на каждом шаге).

В листинге \ref{lst:lab_02.pro} представлен исходный код реализованной программы.
\includelistingpretty
	{lab_02.pro}
	{prolog}
	{Исходный код программы}

\section{Порядок работы системы}

Вопрос: \texttt{subject\_property\_price(''Ivanov'', Type, Price).}

	\footnotesize
	\begin{xltabular}{\textwidth}{|l|X|Y|}
		\hline
		\textbf{№ шага} &
		\textbf{Сравниваемые термы: результат; подстановка, если есть} &
		\textbf{Дальнейшие действия: прямой ход или откат} \\ \hline

		1 & 
		Сравнение:\newline
		\texttt{subject\_property\_price(''Ivanov'', Type, Price) == phonebook(''Ivanov'', ''412314'', address\_struct(''Moscow'', ''Lenina'', 3, 3))} \newline\newline
		Унификация: неудача &
		Прямой ход, переход к следующему предложению \\ \hline

		2 -- 31 & \centering \dots & \dots \\ \hline

		32 &
		Сравнение:\newline
		\texttt{subject\_property\_price(''Ivanov'', Type, Price) == subject\_property\_price(Surname, region, Price)} \newline\newline
		Унификация: успех \newline\newline
		Подстановка:\newline
		\texttt{\{Surname = ''Ivanov'', Type = region, Price = Price\}} & 
		Переход к телу правила. \newline\newline
		Унификация\newline
		\texttt{owner(''Ivanov'', region(Price, \_))} \\ \hline

		33 &
		Сравнение:\newline
		\texttt{owner(''Ivanov'', region(Price, \_)) == phonebook(''Ivanov'', ''412314'', address\_struct(''Moscow'', ''Lenina'', 3, 3))} \newline\newline
		Унификация: неудача &
		Прямой ход, переход к следующему предложению \\ \hline

		34 -- 58 & \centering \dots & \dots \\ \hline

		59 &
		Сравнение:\newline
		\texttt{owner(''Ivanov'', region(Price, \_) == owner(''Ivanov'', region(4325325, 22)} \newline\newline
		Унификация: успех \newline\newline
		Подстановка: \texttt{\{Price = 4325325\}} & 
		Решение найдено, в качестве побочного эффекта получена подстановка \texttt{\{Property = region, Price = 4325325\}} \newline\newline 
		Откат с целью найти все возможные решения.
		Реконкретизация переменной Price.
		Переход к следующему предложению.\\ \hline

		60 &
		Сравнение:\newline
		\texttt{owner(''Ivanov'', region(Price, \_) == owner(''Ivanova'', region(Price, \_)} \newline\newline
		Унификация: неудача &
		Прямой ход, переход к следующему предложению. \\ \hline

		61 -- 66 & \centering \dots & \dots \\ \hline

		67 &
		 &
		Конец базы знаний~--- откат.
		Реконкретизация переменных Surname, Property.
		Переход к следующему предложению относительно шага 32.\\ \hline

		68 &
		Сравнение:\newline
		\texttt{subject\_property\_price(''Ivanov'', Type, Price) == subject\_property\_price(Surname, building, Price)} \newline\newline
		Унификация: успех \newline\newline
		Подстановка:\newline
		\texttt{\{Surname = ''Ivanov'', Type = building, Price = Price\}} & 
		Переход к телу правила. \newline\newline
		Унификация\newline
		\texttt{owner(''Ivanov'', building(Price, \_,))} \\ \hline
		
		69 &
		Сравнение:\newline
		\texttt{owner(''Ivanov'', building(Price, \_,)) == phonebook(''Ivanov'', ''412314'', address\_struct(''Moscow'', ''Lenina'', 3, 3))} \newline\newline
		Унификация: неудача &
		Прямой ход, переход к следующему предложению \\ \hline

		70 -- 91 & \centering \dots & \dots \\ \hline

		92 &
		Сравнение:\newline
		\texttt{owner(''Ivanov'', building(Price, \_,)) == owner(''Ivanov'', building(4214124, address\_struct(''Tver'', ''Lenina'', 1, 1))).} \newline\newline
		Унификация: успех \newline\newline
		Подстановка: \texttt{\{Price = 4214124\}} & 
		Решение найдено, в качестве побочного эффекта получена подстановка \texttt{\{Property = building, Price = 4214124\}} \newline\newline 
		Откат с целью найти все возможные решения.
		Реконкретизация переменной Price.
		Переход к следующему предложению.\\ \hline

		93 &
		Сравнение:\newline
		\texttt{owner(''Ivanov'', building(Price, \_,)) == owner("Petrov", building(235214, address\_struct("Moscow", "Azimova", 12, 13)))} \newline\newline
		Унификация: неудача &
		Прямой ход, переход к следующему предложению \\ \hline

		94 -- 101 & \centering \dots & \dots \\ \hline

		102 &
		 &
		Конец базы знаний~--- откат.
		Реконкретизация переменных Surname, Property.
		Переход к следующему предложению относительно шага 32.\\ \hline

		103 &
		Сравнение:\newline
		\texttt{subject\_property\_price(''Ivanov'', Type, Price) == property\_cost(Surname, car, Cost)} \newline\newline
		Унификация: неудача &
		Прямой ход, переход к следующему предложению \\ \hline

		104 -- 109 & \centering \dots & \dots \\ \hline

		110 &
		&
		Конец базы знаний.
		Завершение работы. \newline\newline
		На вопрос удалось получить ответ <<да>>, поэтому в качестве побочного эффекта возвращено 2 подстановки.
		\\ \hline
	\end{xltabular}