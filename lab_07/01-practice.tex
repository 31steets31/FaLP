\chapter{Практические задания}

\section{Задание}

Необходимо разработать свою программу~--- <<Телефонный справочник и автомобили>>, содержащую следующие домены: 
\begin{enumerate}
	\item <<Телефонный справочник>>:
	\begin{itemize}
		\item фамилия;
		\item номер телефона;
		\item адрес(город, улица, номер дома, номер квартиры);
	\end{itemize}
	\item <<Автомобили>>:
	\begin{itemize}
		\item фамилия владельца;
		\item марка;
		\item цвет;
		\item стоимость;
		\item номер.
	\end{itemize}
\end{enumerate}

\section{Исходный код программы}

В листинге \ref{lst:main.pl} представлен исходный код реализованной программы.
\includelistingpretty
	{main.pl}
	{prolog}
	{Исходный код реализованной программы}

\section{Тестирование}

В листинге \ref{lst:1.pl} представлен результат работы программы в случае поиска одной записи.
\includelistingpretty
	{1.pl}
	{prolog}
	{Результат работы программы в случае поиска одной записи}

В листинге \ref{lst:2.pl} представлен результат работы программы в случае поиска двух записей.
\includelistingpretty
	{2.pl}
	{prolog}
	{Результат работы программы в случае поиска двух записей}
	
В листинге \ref{lst:3.pl} представлен результат работы программы в случае поиска несуществующих записей.
\includelistingpretty
	{3.pl}
	{prolog}
	{Результат работы программы в случае поиска несуществующих записей}

\section{Ответы на вопросы}

\begin{enumerate}
	\item Что собой представляет программа <<Телефонный справочник>> на Prolog?
	\begin{itemize}
		\item Программа <<Телефонный справочник>> на Prolog представляет собой базу знаний и вопрос.
	\end{itemize}
	\item Какова ее структура?
	\begin{itemize}
		\item DOMAINS~--- раздел для описания доменов;
		\item PREDICATES~--- раздел описания предикатов;
		\item CLAUSES~--- раздел описания предложений базы знаний;
		\item GOAL~--- раздел описания цели (вопроса).
	\end{itemize}
	\item Как она реализуется?
	\begin{itemize}
		\item Создается база знаний и вопрос.
	\end{itemize}
	\item Как формируется результат работы программы?
	\begin{itemize}
		\item Осуществляется поиск значений в базе знаний, при которых ответом на поставленный вопрос будет~--- <<Да>>.
	\end{itemize}
\end{enumerate}
