\chapter{Практические задания}

\section{Задание 1}

В листинге \ref{lst:1.lisp} представлена хвостовая рекурсивная функция \texttt{my-reverse}, которая развернет верхний уровень своего списка~-~аргумента \texttt{lst}.
\includelistingpretty
	{1.lisp}
	{lisp}
	{Хвостовая рекурсивная функция \texttt{my-reverse}}

\section{Задание 2}

В листинге \ref{lst:2.lisp} представлена функция, которая возвращает первый элемент списка~-~аргумента, который сам является непустым списком.
\includelistingpretty
	{2.lisp}
	{lisp}
	{Функция, которая возвращает первый элемент списка~-~аргумента}
	
\clearpage

\section{Задание 3}

В листинге \ref{lst:3.lisp} представлена функция, которая выбирает из заданного списка только те числа, которые больше 1 и меньше 10.
\includelistingpretty
	{3.lisp}
	{lisp}
	{Функция, которая выбирает из заданного списка только те числа, которые больше 1 и меньше 10}

\section{Задание 4}

В листинге \ref{lst:4.lisp} представлена функция, которая умножает на заданное число~-~аргумент все числа из заданного списка~-~аргумента, когда:
\begin{enumerate}
	\item все элементы списка~--- числа;
	\item элементы списка~--- любые объекты.
\end{enumerate}
\includelistingpretty
	{4.lisp}
	{lisp}
	{Функция, которая умножает на заданное число~-~аргумент все числа из заданного списка~-~аргумента}

\section{Задание 5}

В листинге \ref{lst:5.lisp} представлена функция \texttt{select-between}, которая из списка~-~аргумента, содержащего только числа, выбирает только те, которые расположены между двумя указанными числами~--- границами~-~аргумента и возвращает их в виде списка (упорядоченного по возрастанию).
\includelistingpretty
	{5.lisp}
	{lisp}
	{Функция \texttt{select-between}}

\section{Задание 6}

В листинге \ref{lst:6.lisp} представлена рекурсивная версия (с именем \texttt{rec-add}) вычисления суммы чисел заданного списка:
\begin{enumerate}
	\item одноуровневого смешанного;
	\item структурированного.
\end{enumerate}
\includelistingpretty
	{6.lisp}
	{lisp}
	{Функция \texttt{rec-add}}

\section{Задание 7}

В листинге \ref{lst:7.lisp} представлена рекурсивная версия с именем \texttt{recnth} функции \texttt{nth}.
\includelistingpretty
	{7.lisp}
	{lisp}
	{Функция \texttt{recnth}}

\section{Задание 8}

В листинге \ref{lst:8.lisp} представлена рекурсивная функция \texttt{allodd}, которая возвращает \texttt{t}, когда все элементы списка нечетные.
\includelistingpretty
	{8.lisp}
	{lisp}
	{Функция \texttt{allodd}}
	
\clearpage

\section{Задание 9}

В листинге \ref{lst:9.lisp} представлена рекурсивная функция, которая возвращает первое нечетное число из списка, возможно, создавая некоторые вспомогательные функции.
\includelistingpretty
	{9.lisp}
	{lisp}
	{Функция, которая возвращает первое нечетное число из списка}

\section{Задание 10}

В листинге \ref{lst:10.lisp} представлена функция, которая получает как аргумент список чисел, а возвращает список квадратов этих чисел в том же порядке.
\includelistingpretty
	{10.lisp}
	{lisp}
	{Функция, которая возвращает список квадратов переданных чисел}