\chapter{Практические задания}

\section{Задание 1}

На рисунке \ref{img:task1} представлены списки в виде списочных ячеек:
\includeimage
	{task1}
	{f}
	{H}
	{1\textwidth}
	{Списки в виде списочных ячеек}

\section{Задание 2}

В листинге \ref{lst:2.lisp} представлены выражения, написанные с использованием функций CAR и CDR и возвращающие 2, 3 и 4 элементы списка.
\includelistingpretty
	{2.lisp}
	{lisp}
	{Выражения, возвращающие 2, 3 и 4 элементы списка}

\section{Задание 3}

Результаты вычисления выражений:
\begin{enumerate}
	\item \texttt{(CAADR '((blue cube) (red pyramid)))}
	
	RED
	\item \texttt{(CDAR '((abc) (def) (ghi)))}
	
	NIL
	\item \texttt{(CADR '((abc) (def) (ghi)))}
	
	(DEF)
	\item \texttt{(CADDR '((abc) (def) (ghi)))}
	
	(GHI)	
\end{enumerate}

\section{Задание 4}

В листинге \ref{lst:4.lisp} представлен результат вычисления выражений:
\includelistingpretty
	{4.lisp}
	{lisp}
	{Результат вычисления выражений}

\section{Задание 5}

В листинге \ref{lst:5.lisp} представлены лямбда-выражения и соответствующие функции:
\includelistingpretty
	{5.lisp}
	{lisp}
	{Лямбда-выражения и соответствующие функции}

\newpage

На рисунке \ref{img:task5} представлены результаты в виде списочных ячеек:
\includeimage
	{task5}
	{f}
	{H}
	{1\textwidth}
	{Результаты в виде списочных ячеек}